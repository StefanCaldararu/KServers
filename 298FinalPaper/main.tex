\documentclass[11pt]{article}
\usepackage{algorithm}
\usepackage{algorithmicx}
\usepackage{algpseudocode}
\usepackage{amssymb}
\usepackage{amsthm}
\usepackage{amsmath}
\usepackage{bm}
\usepackage{bbm}
\usepackage{cite}
\usepackage{color}
\usepackage[inline]{enumitem}
\usepackage[top=1.5in,bottom=1in,right=1in,left=1in]{geometry}
\usepackage{graphicx}
\usepackage{hyperref}
\usepackage{listings}
\usepackage{placeins}
\usepackage{siunitx}
\usepackage{subfig}
\usepackage{todonotes}
\usepackage{wrapfig}
\usepackage{authblk}


\title{An Analysis of K-Server algorithm performance}
\author[1]{Stefan Caldararu}
\author[2]{Marc Renault}
\affil[1]{Undergraduate Student with Department of Computer Science, UW-Madison}
\affil[2]{Professor in the Department of Computer Science, UW-Madison}
\date{\today}                     %% if you don't need date to appear
\setcounter{Maxaffil}{0}
\renewcommand\Affilfont{\itshape\small}


\newcommand{\comment}[1]{{\color{red}\textbf{#1}}}
\newcommand{\CC}{C\nolinebreak\hspace{-.05em}\raisebox{.4ex}{\tiny\bf +}\nolinebreak\hspace{-.10em}\raisebox{.4ex}{\tiny\bf +} }
\newcommand{\KS}{$K$-Servers }
\newcommand{\s}{$\sigma$ }

\theoremstyle{definition}
\newtheorem{definition}{Definition}[section]


% \title{An Analysis of K-Server algorithm performance}
% \author[1]{Stefan Caldararu}
% \author[2]{Marc Renault}
% \affil[1]{Undergraduate Student with Department of Computer Science, UW-Madison}
% \affil[2]{Professor in the Department of Computer Science, UW-Madison}
% \renewcommand\Affilfont{\itshape\small}

\begin{document}
\maketitle
% \begin{titlepage}
% 	\begin{center}
% 		University of Wisconsin-Madison \\
% 		\vfill 
% 		\maketitle
% 		\vspace{0.2in}
% 		{\normalsize Department of Computer Science, University of Wisconsin -- Madison}	
% 		\vfill		
% 		\today		
% 	\end{center}
% \end{titlepage}

% \newpage 
% \vspace{0.1in}
\begin{center}
	Department of Computer Science\\
	University of Wisconsin -- Madison, USA
\end{center}
\vspace{1.5in}
\begin{abstract} 
This project focuses on the \KS problem, and uses multiple analysis methods to gauge performance of different \KS algorithms. Through the first part of the course associated with this project, a literature review was conducted, and a set of algorithms was determined. Following this, an implementation in \CC was written, and practical tests were conducted and later analyzed.
\end{abstract}

{\textbf{Keywords}}: \KS, Online Algorithms, Bijective Analysis

\newpage 

\tableofcontents

\newpage

\section{Introduction}
\label{sec:intro}
\subsection{Outline}
\label{sec:out}
This paper considers the \KS problem, and the practical development of algorithms within a \CC environment. Additionally, we both provide a literature review, and multiple forms of analysis of a variety of algorithms. In sec.~\ref{sec:desc}, we provide a problem description for the \KS problem. In section~\ref{sec:background} a brief literature overview is provided, and then sec.~\ref{sec:AM} describes different performance metrics that can be used when looking at these algorithms. In section~\ref{sec:algDescription}, multiple popular \KS algorithms are described, and a brief list of benefits and drawbacks are provided. We then describe the software implementation for the algorithms in sec.~\ref{sec:implementationDetails}, and some analysis of numerical experiments in sec.~\ref{sec:numExperiments}. Finally, we provide some analysis of results in sec.~\ref{sec:analysis}, and propose future research thrusts in sec.~\ref{sec:conclusion}.

\subsection{Problem Description}
\label{sec:desc}
An instance of the \KS problem can be described by a metric space $M = (X, d)$, a number of servers $k>1$, and an input sequence $\sigma = (r_1, r_2, r_3, ..., r_n)$, where each $r_i$ corresponds to a point in the metric space. Each of the $k$ servers is assigned an initial starting location within the metric space (generally this is assumed to be the first $k$ requests of the input sequence). Following this, The sequence of reqeusts is processed one at a time. When a request comes in, a given algorithm \textit{ALG} must decide on one of the servers to service the request. It must then move said server from it's current location $x$ to the request $r_i$. This incurs a cost of $c = d(x, r_i)$. The goal is to have \textit{ALG} incur the smallest possible cost while servicing all of the requests in the sequence~\cite{OnlineComp1998}.
\\ \\
There are two major distinctions to be made between different classes of algorithms. The first is the classification of a "lazy" algorithm - one that only moves a server in order to service the current request. Non-lazy algorithms will process a request, and then potentially also move other servers preemptively in order to prepare for future requests. It is important to note that for any non-lazy algorithm that performs well, there is a parallel lazy algorithm that performs just as well, if not better. We can describe this algorithm as follows: suppose we have our non-lazy algorithm, \textit{ALG}. We have the algorithm \textit{LAZY} service requests with the same servers that \textit{ALG} services requests. Suppose that \textit{ALG} moves a server from location $x_1$ to location $x_2$ preemptively, and then later services request $r_i$ with this server. \textit{LAZY} will be servicing $r_i$ with the same server, except it will be moving directly from $x_1$ instead of $x_2$. By the triangle inequality, $d(x_1, r_i) \leq d(x_1, x_2) + d(x_2, r_i)$. By applying this principle throughout the request sequence, we will see that the cost of \textit{LAZY} will be as good if not better than that of \textit{ALG}. This shows us that we can create a lazy algorithm from a non-lazy algorithm by maintaining "ghost" locations for servers in relation to how the non-lazy algorithm would use them. Then, we can determine which server the non-lazy algorithm would use, and then service that location with the correct server~\cite{OnlineComp1998}.
\\ \\
The second major distinction is between "online" and "offline" algorithms. An offline algorithm receives the entire request sequence at once, and so as a result is able to make decisions on what server to use for the current request based off of future requests. In contrast, an online algorithm receives the request one at a time, and so is only able to make decisions based off of past requests, the current server configuration, and the current request. While online algorithms are at a severe disadvantage due to this, real world applications often rely on the performance of these algorithms. Applications range from disk access optimization, such as the two headed-disk problem~\cite{OnlineComp1998}, to police or firetruck servicing.

\section{Background}
\label{sec:background}
\subsection{Literature Overview}
\label{sec:lit}
As preliminary research for this project, a literature review was conducted to gauge the current state of research into the \KS problem. First, initial readings included information on online algorithms and competitive analysis proofs, in order to get a solid background on the subject. Future readings progressed to common \KS algorithms, and their performance relative to the competitive ratio~\cite{OnlineComp1998}. Following readings included information about other analysis methods~\cite{MAXMAX2005, bij2016}, which are described in sec.~\ref{sec:AM}. Finally, a practical implementation for the Work Function Algorithm was analyzed~\cite{WFA2009}, and algorithm implementations began, as described in sec.~\ref{sec:algDescription} and sec.~\ref{sec:implementationDetails}.


\subsection{Analysis Methods}
\label{sec:AM}

\subsubsection*{Competitive Analysis}
\label{sec:Comp}
The most popular method for analyzing the performance of online algorithms for a problem is competitive analysis. Here, we assume a "malicious adversary", who is attempting to make our algorithm \textit{ALG} perform as poorly as possible in relation to the performance of the optimal algorithm \textit{OPT}. The malicious adversary is allowed to come up with any finite input, and our competitive ratio is determined by this. If for any finite input, we are able to guaruntee that \textit{ALG} has a cost within a certain ratio $c$ of \textit{OPT} (allowing for a constant additive factor), then we say that our algorithm is $c$-competitive~\cite{OnlineComp1998}. So, if we define $ALG(\sigma)$ as the cost \textit{ALG} incurs while processing request $\sigma$, then we have the following definition: 

\begin{definition}
\label{def:comp}
Algorithm \textit{ALG} is said to be \textbf{\textit{c}-competitive} if for every finite request sequence \s, $ALG(\sigma) \leq c\cdot OPT(\sigma)+\alpha$ for some constant $\alpha$.
\end{definition}

Competitive analysis is in some sense similar to "worst case" analysis, where we try to see how poorly an algorithm will ever perform. This may not always be as useful in practice, as there are algorithms that perform very well on most inputs, but given specific inputs may not be competitive at all. That is, given some value $c$, a finite length input can be found such that the algorithm doesn't satisfy the above definition. Additionally, a lower bound of $k$ has been shown for any online algorithm's competitive ratio. This means that if we are looking at the 3-Servers problem, the best competitive ratio an online algorithm can achieve is 3~\cite{OnlineComp1998}. 

\subsubsection*{Direct Analysis}
\label{sec:Direct}
Direct analysis is a similar techinque to competitive analysis, in that we directly compare the performance of \textit{ALG} to the performance of \textit{OPT} on a request. Rather than looking across all request sequences of any finite length, we determine a specified length for our request sequence. Then, we look directly at the ratio of \textit{ALG}'s to \textit{OPT}'s performance for each input, and take the largest such ratio. It is important to note that as the length of our input approaches infinity, our direct analysis ratio will aproach the competitive ratio of our algorithm.

\begin{definition}
    \label{def:direct}
    Algorithm \textit{ALG} has a direct analysis ratio of $c$ if for every input \s of length $n$, $ALG(\sigma) \leq c\cdot OPT(\sigma)$.
\end{definition}

\subsubsection*{Max/Max Ratio}
\label{sec:MaxMax}
For the Max/Max ratio, we are comparing the worst case of each algorithm to each other. In practice, this makes sense to do on finite sets of input sequences. Here, we will take the highest cost of \textit{ALG} and the highest cost of \textit{OPT}, and this ratio will be our performance metric~\cite{MAXMAX2005}. 

\subsubsection*{Bijective Analysis}
\label{sec:Bij}
The bijective ratio is similar to using direct analysis, except we allow for a bijection between the two data sets. So, we look at the input space of all inputs of length $n$, denoted $I_n$. If there exists a bijetion $\pi:I_n \rightarrow I_n$ such that $ALG(\sigma) \leq c\cdot OPT(\pi(\sigma))$, then  we say that the bijective ratio between \textit{ALG} and \textit{OPT} is $c$~\cite{bij2016}. Therefore, we obtain def.~\ref{def:bij}. 

\begin{definition}
    \label{def:bij}
    For a given input space $I_n$, we say that algorithm \textit{A} has bijective ratio $c$ with respect to algorithm \textit{B} if there exists a bijection $\pi:I_n \rightarrow I_n$ such that $A(\sigma) \leq c\cdot B(\pi(\sigma))$, $\forall \sigma \in I_n$. 
\end{definition}

It is important to note that this definition doesn't only compare the ratio of an algorithm to the optimal, but can also be used to compare between two different online algorithms. This can be used to prove an online algorithms optimality~\cite{bij2016}, and allows for some interesting test-case results found in sec.~\ref{sec:analysis}.


\section{Algorithm Description}
\label{sec:algDescription}

\subsection{Algorithm Derivation}
Write about what algorithms were implemented, how the different algorithms work... etc.

\section{Software implementation details and API}
\label{sec:implementationDetails}
This program is implemented in C++, and is available publicly at https://github.com/StefanCaldararu/KServers. There is a main program defined in the "csv parser" file, which is intended to take a csv file as input, and will output a corresponding csv file. Additionally, there is an implementation of this parser that is able to run multiple inputs in parallel, and is designed to run with 16 independant threads. The input file should be formatted with the first line specifying which algorithms should be run. There will either be a 1, or a 0 corresponding to whether or not the algorithm is being run for this input. The following input line is a single integer, describing how many input metric spaces there will be. The following is repeated for each metric space. First, the size of the metric space, $n$ is defined. Following that, $n$ lines of length $n$ are written, corresponding to the cost to get from the row index to the column index. Following this, the number of inputs is specified, and the number of servers for this problem is specified on seperate lines. Finall, inputs are written on individual lines. The output file is formatted such that the input is printed, and then each algorithms output is printed for each input.
\\ \\
There are a couple of notable class definitions in this application that help the algorithms run. First, there is a metric space class, which holds an adjacency matrix of cost between nodes. This also has information about the server configuration, number of servers, and has methods allowing the main program to move the servers around and edit the metric space. There is additionally an Algorithm super class, which each algorithm derives the majority of its methods from. Finally, the algorithms 
\textit{OPT} and \textit{WFA} use the class "Mcfp" in order to compute their Min Cost Max Flow problems. This allows them to generate network flow graphs, compute their costs, and for \textit{WFA}, determine which server to move.
\\ \\
Finally, there is a set of programs designed to generate input files. These files can generate inputs of lines or circles, with any number of nodes and servers.

\section{Numerical Experiments}
\label{sec:numExperiments}
In this section, we will present some of the numerical results from tests that were run. These are entirely experiment based, and there are no proof based results yet. In section~\ref{sec:analysis}, we will describe some of the interesting qualities of the results presented. Proof of these results will be a part of the future works. 
\\ \\
Data on the bijective ratios between \textit{OPT}, \textit{GREEDY}, and \textit{WFA} are provided, as these were found to be the more intersting results. Table~\ref{tab:tests} shows which tests were run, and assigns each test an input number. Table~\ref{tab:results} shows the results of the bijective analysis for each input number. Tests were run on a line metric space, and circle metric space. For all tests, points are distributed with uniform distance.

\begin{table}[!htb]
    \begin{minipage}{.5\linewidth}
      \centering
      \begin{tabular}{|c|c|c|c|c|c|}
        \hline
        Input Number & Space Type & Space Size & Servers & Input Length & Number of Inputs\\
        \hline
        1 & Line & 6 & 2 & 8 & ALL \\
        \hline
        2 & Line & 8 & 2 & 7 & ALL \\
        \hline
        3 & Line & 8 & 3 & 7 & ALL\\
        \hline
        4 & Line & 10 & 3 & 8 & ALL\\
        \hline
        5 & Line & 10 & 3 & 20 & 10,000,000\\
        \hline
        6 & Line & 10 & 3 & 100 & 1,000,000\\
        \hline 
        7 & Circle & 6 & 2 & 8 & ALL\\
        \hline
        8 & Circle & 8 & 3 & 8 & ALL\\
        \hline
        9 & Circle & 10 & 2 & 6 & ALL\\
        \hline
        10 & Circle & 10 & 3 & 6 & ALL\\
        \hline
        11 & Circle & 20 & 3 & 6 & ALL \\
        \hline
    \end{tabular}
        \caption{Tests}
        \label{tab:tests}
    \end{minipage}%
    \\
    \begin{minipage}{.5\linewidth}
        \centering
        \begin{tabular}{|c|c|c|c|}
          \hline
          Input Number & GRE/OPT Bij. & WFA/OPT Bij. & WFA/GRE Bij.\\
          \hline
          1 & 3/2 & 3/2 & 4/3\\
          \hline
          2 & 3/2 & 1.55 & 4/3\\
          \hline
          3 & 3/2 & 1.6 & 4/3 \\
          \hline
          4 & 3/2 & 1.59 & 4/3 \\
          \hline
          5 & 1.319 & 1.308 & 1.03 \\
          \hline
          6 & 1.447 & 1.486 & 1.2 \\
          \hline
          7 & 3/2 & 3/2 & 7/6 \\
          \hline
          8 & 3/2 & 3/2 & 4/3 \\
          \hline
          9 & 3/2 & 3/2 & 7/6 \\
          \hline
          10 & 3/2 & 3/2 & 4/3 \\
          \hline
          11 & 3/2 & 3/2 & 4/3 \\
          \hline

      \end{tabular}
          \caption{Results}
          \label{tab:results}
    \end{minipage} 
\end{table}

\section{Analysis}
\label{sec:analysis}
\subsection{Analysis}
Describe the results from the data that was collected, what is interesting, different results...

\section{Conclusion}
\label{sec:conclusion}
Write something here...


\bibliographystyle{ieeetr}
\bibliography{refs.bib}


\end{document}          