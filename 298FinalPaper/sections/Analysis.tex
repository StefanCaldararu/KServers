While data was also collected for bijective ratios between the other algorithms, as well as Max/Max ratios and direct analysis ratios, the data presented is the most notable. \textit{GREEDY} has already been show to be non-competitive, and competitive ratios for \textit{DC} and \textit{KS} have already been proven. For \textit{WFA}, a competitive ratio of $2k-1$ has been proven, and no input with competitive ratio worse than $k$ has been found. This makes the Direct Analysis rather uninteresting. Additionally, the Max/Max ratio on small inputs such as these doesn't seem to provide much insight into patterns, or the true performance of the algorithms.
\\ \\
The first thing to note is that running these algorithms on a small subset of random inputs of a certain length seems to provide little to no insight into the algorithms true performace. Both \textit{GREEDY} and \textit{WFA} perform well - if not optimally - on a large subset of the inputs. As our input lengths increase, the number of possible inputs increases exponentially. To demonstrate the phenomena we observe, suppose for example that \textit{GREEDY} and \textit{WFA} perform optimally on 90\% of inputs, and we are able to run inputs 1,000,000,000 of any length on a metric space with 10 points in reasonable time (this is not true, as our computation time increase drastically with regard to input length due to the WFA). So, wecan run all inputs of length 9 within this time. If we want to run inputs of length 10, we are only able to run 10\% of the inputs. If we want to run inputs of length 12, we can only run 0.1\% of the inputs! The probability that all of these inputs will land within the 90\% where the algorithms perform optimally increases exponentially as our input length increases, making our bijective ratio approach 1.
\\ \\
Therefore, we can only start to draw interesting conclusions from the input parameters where we can run any possible input sequence. \comment{is "greedy optimal on circle"} consistent with the 1.5?
\\ \\
The final interesting result is with respect to the comparison between \textit{WFA} and \textit{GREEDY}. The bijection in the other direction is not shown, as it is always 1 (meaning that \textit{GREEDY} always performs as well if not better than \textit{WFA}, bijectively). The fact that the bijective ratio shown is not 1 