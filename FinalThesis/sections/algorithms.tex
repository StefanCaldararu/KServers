
\subsection{Random Algorithm}
\label{sec:rand}
The random algorithm \textit{RAND} is the most basic of all of the online algorithms, and has little practical use. If the request is not currently covered by a server, then \textit{RAND} randomly selects one of it's servers, and moves the selected server to the service point. This algorithm is mostly just used as a baseline, as we would hope that none of the other algorithms will perform worse than random.

\subsection{Greedy Algorithm}
\label{sec:greedy}
The greedy algorithm \textit{GREEDY} is also a computationally inexpensive online algorithm, but which has many practical uses. This algorithm checks the distance that each server would have to travel to get to the service point, and selects the server which would incur the smallest cost. While this algorithm has very good performance for the majority of practical application inputs, it is surprisingly not a competitive algorithm~\cite{OnlineComp1998}. \textit{GREEDY} is one of the primary focuses of this study, as it is often looked over due to it's non-competitiveness, but still performs very well in practice.

\subsection{Optimal Algorithm}
\label{sec:OPT}
Any optimal offline algorithm \textit{OPT} is simply defined as an algorithm that will have the smallest possible cost for any request sequence. Computationally, the fastest implementations leverage a reduction to a Min Cost Max Flow problem. This reduction can be further studied in~\cite{WFA2009}. This still ends up being a good bit more computationally expensive than the previous algorithms, but is needed to be used as a metric to compare algorithms against, as most algorithms are compared to the optimal when determining their strenghts.

\subsection{Work-Function Algorithm}
\label{sec:WFA}
The Work-Function Algorithm \textit{WFA} is considered to be one of the most promising algorithms in terms of the competitive ratio, as it has been proven to be $(2k-1)$-competitive, and is believed to be $k$-competitive. Given a certain starting server configuration, current configuration, and previous input request sequence, \textit{WFA} determines which server an optimal algorithm would move, while also ensuring that $k-1$ of the servers end up in the current server configuration. This final server is then used to service the current request. A similar reduction to the \textit{OPT} computation can be used to find this server~\cite{WFA2009}. This means that the \textit{WFA} must compute a Min Cost Max Flow for each request, making it much more expensive than \textit{OPT}, and all of the other algorithms. Additionally, it is also worth noting that this is not a finite memory algorithm, as the WFA must remember all of the previous requests~\cite{MAXMAX2005}. 

\subsection{Double Coverage Algorithm}
\label{sec:DC}
The Double Coverage algorithm \textit{DC} is a $k$-competitive algorithm on the line. If the request is to the left of all of our current server locations, then we just move the left most server to service the request. If the server is to the right of all of the current locations, we use the rightmost server. Otherwise, the request will be between two servers. In this case, we move the two closest servers towards the reqeust at the same rate, until one of the servers reaches the request location. A proof of \textit{DC}'s competitiveness can be found in~\cite{OnlineComp1998}. It is also important to note that \textit{DC} is not a lazy algorithm, as it does move more than one server at a time. Additionally, generalizations of this algorithm can be used on metric spaces other than the line, such as trees.

\subsection{$k$-Centers Algorithm}
\label{sec:KC}
The $k$-Centers algorithm divides the metric space into $k$ sections, and assigns each server a section of the space to operate in. Implementations of this algorithm can either have the server return to the center of it's operating space after servicing the request in a non-lazy fashion, or just remember the bounds of each operating space, and service each request with the appropriate server~\cite{bij2016}.