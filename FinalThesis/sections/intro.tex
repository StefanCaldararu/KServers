\subsection{Outline}
\label{sec:out}
In this paper, we consider the \KS problem in a variety of contexts. First, we provide a description of the \KS problem, and an in-depth proof of the $2k-1$ competitiveness of the Work Function Algorithm (WFA). Following this a literature review is provided, focusing on the Unifying Potential for the WFA~\cite{unifyingPotential2021}. We then have an analysis of~\cite{unifyingPotential2021}, and provide some thoughts into a potential extension of this work for caterpillar graphs. Following this, we transition to an analysis of other algorithms focusing on the bijective analysis of these algorithms, and describe a \CC environment provided for practical testing of a variety of these algorithms. We provide some experimental analysis of the algorithms within the testing suite, extending the work done in~\cite{independantStudy2023}.

\subsection{Problem Description}
\label{sec:desc}
An instance of the \KS problem can be described by a metric space $M = (X, d)$, a number of servers $k>1$, and an input sequence $\sigma = (r_1, r_2, r_3, ..., r_n)$, where each $r_i$ corresponds to a point in the metric space. Each of the $k$ servers is assigned an initial starting location within the metric space (generally this is assumed to be the first $k$ requests of the input sequence). Following this, The sequence of reqeusts is processed one at a time. When a request comes in, a given algorithm \textit{ALG} must decide on one of the servers to service the request. It must then move said server from it's current location $x$ to the request $r_i$. This incurs a cost of $c = d(x, r_i)$. The goal is to have \textit{ALG} incur the smallest possible cost while servicing all of the requests in the sequence~\cite{OnlineComp1998}.
\\ \\
There are two major distinctions to be made between different classes of algorithms. The first is the classification of a "lazy" algorithm - one that only moves a server in order to service the current request. Non-lazy algorithms will process a request, and then potentially also move other servers preemptively in order to prepare for future requests. It is important to note that for any non-lazy algorithm that performs well, there is a parallel lazy algorithm that performs just as well, if not better. We can describe this algorithm as follows: suppose we have our non-lazy algorithm, \textit{ALG}. We have the algorithm \textit{LAZY} service requests with the same servers that \textit{ALG} services requests. Suppose that \textit{ALG} moves a server from location $x_1$ to location $x_2$ preemptively, and then later services request $r_i$ with this server. \textit{LAZY} will be servicing $r_i$ with the same server, except it will be moving directly from $x_1$ instead of $x_2$. By the triangle inequality, $d(x_1, r_i) \leq d(x_1, x_2) + d(x_2, r_i)$. By applying this principle throughout the request sequence, we will see that the cost of \textit{LAZY} will be as good if not better than that of \textit{ALG}. This shows us that we can create a lazy algorithm from a non-lazy algorithm by maintaining "ghost" locations for servers in relation to how the non-lazy algorithm would use them. Then, we can determine which server the non-lazy algorithm would use, and then service that location with the correct server~\cite{OnlineComp1998}.
\\ \\
The second major distinction is between "online" and "offline" algorithms. An offline algorithm receives the entire request sequence at once, and so as a result is able to make decisions on what server to use for the current request based off of future requests. In contrast, an online algorithm receives the request one at a time, and so is only able to make decisions based off of past requests, the current server configuration, and the current request. While online algorithms are at a severe disadvantage due to this, real world applications often rely on the performance of these algorithms. Applications range from disk access optimization, such as the two headed-disk problem~\cite{OnlineComp1998}, to police or firetruck servicing.