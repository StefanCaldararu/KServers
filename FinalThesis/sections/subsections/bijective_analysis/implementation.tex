In this section, we describe the software package developed as part of this work. The package is written in C++, and has various levels of performance capabilities. It can additionally be run using the Lemon Graph Library~\cite{lemon}, or with purely internal data structure implementations. The various implementaiotns allow users to run the various algorithms described in sec.~\ref{sec:algDescription} either with a single thread, or under a producer-consumer framework for parallel processing. Additionally, a finite memory implementation is provided, and a memoization approach on the input sequence is implemented allowing for increased efficiency. Finally, example code for running the software on a High Throughput Cluster using Condor~\cite{htcondor} is provided.

\subsubsection*{Basic structures}

The most basic structure in the software package is the \texttt{Mspace} class. This class uses a \texttt{std::vector} to store the graph as an adjacency matrix of distances between nodes. Functions such as \texttt{setSize}, \texttt{getSize}, \texttt{getDistance}, and \texttt{setDistance} allow for easy manipulation of the metric space.

\texttt{getInput} and \texttt{writeOutput} classes are written to help maintain RAII standards, and allow for easy reading and writing of input and output files. 

An abstract \texttt{Alg} class is provided as a framework for each of the various algorithms to build off of. This class defines all basic functionality for an algorithm to run, and has a virtual function \texttt{run} that must be implemented by each algorithm, which takes an input request sequence and returns the cost incurred by the algorithm. It contains a \texttt{Mspace} object, and stores the current configuration of the servers in two forms: a \texttt{config} vector which stores the integer location of each server, as well as a \texttt{coverage} vector that stores whether or not each location in the metric space is covered by a server. This allows for efficient checking of whether or not a location is covered by a server, as well as efficient access to where the servers are located. \texttt{setServers} and \texttt{setGraph} functions are provided to allow for easy initialization of the algorithm, and a \texttt{moveServer} function is provided for easy manipulation of the server configuration.

