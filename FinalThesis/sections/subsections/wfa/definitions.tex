
We begin with a couple of notation definitions for clarity.

\begin{definition}
    A request sequence $\sigma$ can be broken up into sub-parts, where $\sigma_i$ represents the first $i$ requests from within the request sequence
\end{definition}

\begin{definition}
    A set $C$ of k-points from within the metric space $M$ that the $WFA$ is operating in is called a \textbf{configuration}, representing the locations of the servers. Additionally, the initial configuration before any requests have been processed is denoted $C_0$, and the configuration after the WFA has operated on the first $i$ requests is denoted $C_{\sigma_i}$. Additionally, $C$ can be used to denote an arbitrary configuration after servicing some request sequence $\sigma_i$, in which case $C'$ denotes the configuration after servicing the next request, i.e. servicing the request sequence $\sigma_{i+1}$.
\end{definition}

\begin{definition}
    The distance between two configurations $A$ and $B$ can be computed as the minimum weight matching between the two sets, denoted $D(A, B)$.
\end{definition}

\begin{definition}
    The \textbf{work function}, denoted $w_\sigma(C)$ computes the minimum value required to begin in configuration $C_0$, service all requests in $\sigma$, and end with servers in configuration $C$. The work function has similar notation to configurations($w_{\sigma_i}$, $w$, and $w'$), except the work function value on an empty request sequence is denoted $w_\emptyset$. Additionally, the work function can be computed as follows:
    \begin{equation*}
        \begin{gathered}
            w_\emptyset(C) = D(C_0, C) \\
            w_\sigma(C) = min_{x \in C} \{ w(C - x + r) + d(x, r)\}
        \end{gathered}
    \end{equation*}
\end{definition}

\begin{definition}
    The \textbf{work function algorithm} moves the server currently at point $s$ at each step, incurring a cost $d(s,r)$. This server is determined as follows:
    \begin{equation*}
        s = argmin_{x \in C} \{ w(C-x+r) + d(x,r)\}
    \end{equation*}
\end{definition}