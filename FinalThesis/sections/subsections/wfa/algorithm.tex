This section focuses on an in-depth analysis of the WFA. There are two ways of thinking about the WFA's decision-making process. WFA can be thought of as a balancing between retrospective-$OPT$, and a greedy algorithm. It tries to balance between the decisions $OPT$ would have made up until this point, assuming the current request is the last one that will be made, with a greedy algorithm that attempts to move the least distance possible at the current time step. This helps the algorithm maintain a configuration that is similar to $OPT$, without moving servers great distances from the current configuration.

Additionally, $WFA$ can be thought of from how it's implementation works. $WFA$ makes lazy decisions to move servers, so only moves one server at each step. As described previously, the $WFA$ will attempt to determine the server to move which would have the minimum cost for starting in an initial configuration, and ending with $k-1$ servers in the same locations they are currently in. Then, it will service the request with the server that does not stay in the initial configuration.