In this section, we present a few additional methods for analysis on algorithms other than competitive analysis. These methods are useful for comparing algorithms to each other, and can be used to demonstrate an algorithms efficiency outside of the "worst-case" considered by competitive analysis. Competitive analysis considers all requests of finite length. For each of these methods described below, we take an experimental-based approach. That is, each analysis method will be considerind the performance of an algorithm on a given request sequence length, and so can therefore be directly calculated in practice.

\subsubsection*{Direct Analysis}
\label{sec:Direct}
Direct analysis is a similar techinque to competitive analysis, in that we directly compare the performance of \textit{ALG} to the performance of \textit{OPT} on a request. Rather than looking across all request sequences of any finite length, we determine a specified length for our request sequence. Then, we look directly at the ratio of \textit{ALG}'s to \textit{OPT}'s performance for each input, and take the largest such ratio. It is important to note that as the length of our input approaches infinity, our direct analysis ratio will aproach the competitive ratio of our algorithm.

\begin{definition}
    \label{def:direct}
    Algorithm \textit{ALG} has a direct analysis ratio of $c$ if for every input \s of length $n$, $ALG(\sigma) \leq c\cdot OPT(\sigma)$.
\end{definition}

\subsubsection*{Max/Max Ratio}
\label{sec:MaxMax}
For the Max/Max ratio, we are comparing the worst case of each algorithm to each other. In practice, this makes sense to do on finite sets of input sequences. Here, we will take the highest cost of \textit{ALG} and the highest cost of \textit{OPT}, and this ratio will be our performance metric~\cite{MAXMAX2005}. 

\subsubsection*{Bijective Analysis}
\label{sec:Bij}
The bijective ratio is similar to using direct analysis, except we allow for a bijection between the two data sets. So, we look at the input space of all inputs of length $n$, denoted $I_n$. If there exists a bijetion $\pi:I_n \rightarrow I_n$ such that $ALG(\sigma) \leq c\cdot OPT(\pi(\sigma))$, then  we say that the bijective ratio between \textit{ALG} and \textit{OPT} is $c$~\cite{bij2016}. Therefore, we obtain def.~\ref{def:bij}. 

\begin{definition}
    \label{def:bij}
    For a given input space $I_n$, we say that algorithm \textit{A} has bijective ratio $c$ with respect to algorithm \textit{B} if there exists a bijection $\pi:I_n \rightarrow I_n$ such that $A(\sigma) \leq c\cdot B(\pi(\sigma))$, $\forall \sigma \in I_n$. 
\end{definition}

It is important to note that this definition doesn't only compare the ratio of an algorithm to the optimal, but can also be used to compare between two different online algorithms. This can be used to prove an online algorithms optimality~\cite{bij2016}, and allows for some interesting test-case results found in sec.~\ref{sec:analysis}.

\subsubsection*{Average Case Analysis}
\label{sec:Avg}
The average case analysis compares the average cost of two algorithms across a set of request sequences to each other. This is useful for comparing algorithms in practice, and additionally allows us to compare two algorithsm without storing their output for every input, which helps with both memory usage and storage usage.

\begin{definition}
    \label{def:avg}
    For a given input space $I_n$, we say that algorithm \textit{A} has an average case ratio $c$ with respect to algorithm \textit{B} if the average cost of \textit{A} is at most $c$ times the average cost of \textit{B} over all inputs of length $n$.
\end{definition}