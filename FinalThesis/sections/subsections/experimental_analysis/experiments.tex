Experimental data collected fits into the above categories of analysis methods. First, we focus on Bijective analysis comparisons between $WFA$, $GREEDY$, and $OPT$. The three algorithms are compared against each other on the line metric space as well as the circle metric space, and experimental results are discussed. Average case analysis is discussed for all algorithms, specifically on the reduced caterpillar graph. 

In regards to competitive ratios, we have demonstrated that $WFA$ is $2k-1$-competitive on all metric spaces, and it is believed to be $k$-competitive. Contrarily, we prove the following lemma for $GREEDY$:

\begin{lemma}
    $GREEDY$ is not a competitive algorithm. i.e. there exists no $c$ such that $GREEDY$ is $c$-competitive.
\end{lemma}

\begin{proof}
    We must show that given any $c$, we can generate a metric space and input request sequence where $GREEDY$ has a competitive ratio of at least $c$. We consider the following example:
    \begin{figure}[H]
        \centering
        \includegraphics[width=0.5\textwidth]{images/line.png}
        \caption{An example metric space for $GREEDY$'s non-competitiveness}
    \end{figure}
    Suppose we have $k = 2$, and both $GREEDY$ and $OPT$ begin with servers at $b$ and $c$. We begin by requesting $a$, and then proceed to alternate requests between $b$ and $c$. $OPT$ will first service $a$ with the server at $b$, and then move this server back to $b$ incurring a total cost of 10, and from there never increasing. $GREEDY$ will service $a$ with the server at $b$, but then proceed to move the server at $c$ back and forth between $b$ and $c$, in accordance with its policy. Therefore, $GREEDY$ will incur a cost of 1 each time a new request is processed, forever increasing. It is clear that the competitive ratio of $GREEDY$ is unbounded, and therefore $GREEDY$ is not a competitive algorithm.
\end{proof}

\subsubsection*{Bijective Analysis}
While from the worst-case perspective of the competitive ratios $GREEDY$ is worse than $WFA$, Bijective Analysis tells a different story. In this section, we provide some experimental results for the bijective ratios between $WFA$ and $GREEDY$, supporting the results shown in~\cite{bij2016} that $GREEDY$ is actually a bijectively optimal online algorithm.


\begin{table}[!htb]
    \begin{minipage}{.5\linewidth}
      \centering
      \begin{tabular}{|c|c|c|c|c|c|}
        \hline
        Input ID & Space Type & Space Size & Servers & Input Length & Number of Inputs\\
        \hline
        1 & Line & 6 & 2 & 8 & ALL \\
        \hline
        2 & Line & 8 & 2 & 7 & ALL \\
        \hline
        3 & Line & 8 & 3 & 7 & ALL\\
        \hline
        4 & Line & 10 & 3 & 8 & ALL\\
        \hline
        5 & Line & 10 & 3 & 20 & 10,000,000\\
        \hline
        6 & Line & 10 & 3 & 100 & 1,000,000\\
        \hline 
        7 & Circle & 6 & 2 & 8 & ALL\\
        \hline
        8 & Circle & 8 & 3 & 8 & ALL\\
        \hline
        9 & Circle & 10 & 2 & 6 & ALL\\
        \hline
        10 & Circle & 10 & 3 & 6 & ALL\\
        \hline
        11 & Circle & 20 & 3 & 6 & ALL \\
        \hline
    \end{tabular}
        \caption{Tests}
        \label{tab:tests}
    \end{minipage}%
    \\
    \begin{minipage}{.5\linewidth}
        \centering
        \begin{tabular}{|c|c|c|c|}
          \hline
          Input ID & GRE/OPT Bij. & WFA/OPT Bij. & WFA/GRE Bij.\\
          \hline
          1 & 3/2 & 3/2 & 4/3\\
          \hline
          2 & 3/2 & 1.55 & 4/3\\
          \hline
          3 & 3/2 & 1.6 & 4/3 \\
          \hline
          4 & 3/2 & 1.59 & 4/3 \\
          \hline
          5 & 1.319 & 1.308 & 1.03 \\
          \hline
          6 & 1.447 & 1.486 & 1.2 \\
          \hline
          7 & 3/2 & 3/2 & 7/6 \\
          \hline
          8 & 3/2 & 3/2 & 4/3 \\
          \hline
          9 & 3/2 & 3/2 & 7/6 \\
          \hline
          10 & 3/2 & 3/2 & 4/3 \\
          \hline
          11 & 3/2 & 3/2 & 4/3 \\
          \hline

      \end{tabular}
          \caption{Results}
          \label{tab:results}
    \end{minipage} 
\end{table}

The first thing to note is that running these algorithms on a small subset of random inputs of a certain length seems to provide little to no insight into the algorithms true performace. Both $GREEDY$ and $WFA$ perform well - if not optimally - on a large subset of the inputs. As our input lengths increase, the number of possible inputs increases exponentially. To demonstrate the phenomena we observe, suppose for example that $GREEDY$ and $WFA$ perform optimally on 90\% of inputs, and we are able to run 1,000,000,000 inputs of any length on a metric space with 10 points in reasonable time (this is not true, as our computation time increase drastically with regard to input length due to the WFA). So, we can run all inputs of length 9 within this time. If we want to run inputs of length 10, we are only able to run 10\% of the inputs. If we want to run inputs of length 12, we can only run 0.1\% of the inputs! The probability that all of these inputs will land within the 90\% where the algorithms perform optimally increases exponentially as our input length increases, making our bijective ratio approach 1. This is demonstrated in input ID 5 and 6. For both of these we observe smaller experimental bijective ratios than on the inputs where we are able to run all of the possible request sequences. Therefore, we can only start to draw interesting conclusions from the input parameters where we can run every possible input sequence.

The final interesting result is with respect to the comparison between $WFA$ and $GREEDY$. The bijection in the other direction is not shown, as it is always 1 (meaning that $GREEDY$ always performs as well if not better than $WFA$, bijectively). These experimental results support the proofs in~\cite{bij2016} that $GREEDY$ is a bijective optimal algorithm. The fact that the bijective ratio shown is not 1 suggests some interesting results. If both of these bijections could be proven, then this would show that it is better to use $GREEDY$ rather than $WFA$ (a computationally much more expensive algorithm), across a uniform distribution of inputs. For every input where $GREEDY$ performs poorly, there would be an input where $WFA$ performs just as poorly if not worse. This is an interesting result, and suggests that $GREEDY$ is a better algorithm to use in practice.

\subsubsection*{Direct Analysis and Average Case Analysis}
Direct analysis and Average case experiments have been run on the reduced caterpillar graph, and the results are shown in the table below. We provide experiments for a variety of input lengths and metric space sizes, with 2 and 3 servers. For the $WFA$, the direct analysis ratios for all inputs reflect the current belief that it is a $k$-competitive algorithm. It is interesting to note that $GREEDY$ has the same direct analysis ratio for metric spaces of size 12 and 15 for both 2 and 3 servers.

The average case ratios for $GREEDY$ are consistently better than those for $WFA$. This supports the claims made above, that in general for a uniform distribution of inputs, $GREEDY$ is a better algorithm to use in practice. It is worth noting that the current input requests do consider all possible inputs, so a sequence of requests to the same point multiple times in a row is possible. It may be intersting to analyze this problem on a reduced input space, where only a subset of "interesting" inputs are considered.

Unfortunately, average case analysis results were not collected for the 5th input. 

\begin{table}[!htb]
    \begin{minipage}{.5\linewidth}
      \centering
      \begin{tabular}{|c|c|c|c|c|c|}
        \hline
        Input ID & Space Size & Servers & Input Length & Number of Inputs\\
        \hline
        1 & 12 & 2 & 6 & ALL \\
        \hline
        2 & 12 & 3 & 6 & ALL \\
        \hline
        3 & 15 & 2 & 6 & ALL \\
        \hline
        4 & 15 & 3 & 6 & ALL \\
        \hline
        5 & 6 & 3 & 10 & ALL\\
        \hline
        6 & 6 & 3 & 12 & ALL\\
        \hline
        7 & 9 & 3 & 11 & ALL\\
        \hline
    \end{tabular}
        \caption{Tests for Reduced Caterpillar Graphs}
        \label{tab:tests1}
    \end{minipage}%
    \\
    \begin{minipage}{.5\linewidth}
        \centering
        \begin{tabular}{|c|c|c|c|c|}
          \hline
          Input ID & WFA Direct & GREEDY Direct & WFA Avg & GREEDY Avg\\
          \hline
          1 & 2 & 11/3 & 0.783765 & 0.55331442\\
          \hline
          2 & 3 & 3 & 0.845332 & 0.69159289\\
          \hline
          3 & 2 & 11/3 & 0.76000421 & 0.54207148\\
          \hline
          4 & 3 & 3 & 0.870354 & 0.59062018 \\
          \hline
          5 & 3 & 5 &  &  \\
          \hline
          6 & 3 & 6 & 1.50743743999 & 1.236367319548 \\
          \hline
          7 &  &  &  & \\
          \hline

      \end{tabular}
          \caption{Direct Analysis Results}
          \label{tab:results1}
    \end{minipage} 
\end{table}