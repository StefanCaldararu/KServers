\documentclass[11pt]{article}
\usepackage{algorithm}
\usepackage{algorithmicx}
\usepackage{algpseudocode}
\usepackage{amssymb}
\usepackage{amsthm}
\usepackage{amsmath}
\usepackage{bm}
\usepackage{bbm}
\usepackage{cite}
\usepackage{color}
\usepackage[inline]{enumitem}
\usepackage[top=1.5in,bottom=1in,right=1in,left=1in]{geometry}
\usepackage{graphicx}
\usepackage{hyperref}
\usepackage{listings}
\usepackage{placeins}
\usepackage{siunitx}
\usepackage{subfig}
\usepackage{todonotes}
\usepackage{wrapfig}
\usepackage{authblk}
\usepackage[english]{babel}
\newtheorem{theorem}{Theorem}[section]
\newtheorem{corollary}{Corollary}[theorem]
\newtheorem{lemma}[theorem]{Lemma}


\title{Title}
\author[1]{Stefan Caldararu}
\affil[1]{Undergraduate Student with Department of Computer Science, UW-Madison}
\date{\today}                     %% if you don't need date to appear
\setcounter{Maxaffil}{0}
\renewcommand\Affilfont{\itshape\small}


\newcommand{\comment}[1]{{\color{red}\textbf{#1}}}
\newcommand{\CC}{C\nolinebreak\hspace{-.05em}\raisebox{.4ex}{\tiny\bf +}\nolinebreak\hspace{-.10em}\raisebox{.4ex}{\tiny\bf +} }
\newcommand{\KS}{$K$-Server }
\newcommand{\s}{$\sigma$ }

\theoremstyle{definition}
\newtheorem{definition}{Definition}[section]


% \title{An Analysis of K-Server algorithm performance}
% \author[1]{Stefan Caldararu}
% \author[2]{Marc Renault}
% \affil[1]{Undergraduate Student with Department of Computer Science, UW-Madison}
% \affil[2]{Professor in the Department of Computer Science, UW-Madison}
% \renewcommand\Affilfont{\itshape\small}

\begin{document}
\maketitle
% \begin{titlepage}
% 	\begin{center}
% 		University of Wisconsin-Madison \\
% 		\vfill 
% 		\maketitle
% 		\vspace{0.2in}
% 		{\normalsize Department of Computer Science, University of Wisconsin -- Madison}	
% 		\vfill		
% 		\today		
% 	\end{center}
% \end{titlepage}

% \newpage 
% \vspace{0.1in}
\begin{center}
	Department of Computer Science\\
	University of Wisconsin -- Madison, USA
\end{center}
\vspace{1.5in}
\begin{abstract} 
	In this work, we focus on the \KS problem, a well-known problem in the field of online algorithms. We provide an overview of various algorithms used to solve this problem, and provide a detailed analysis of the Work Function Algorithm~\cite{KS1990, OnlineComp1998}. We then transition to the Unifying Potential, a recently proposed method of analysis for the \KS problem~\cite{unifyingPotential2021}. We provide an overview of the definitions and proofs used in this method, and attempt to extend their methods to Reduced Caterpillar Graphs. Finally, we provide an open-source software framework for testing and comparing these algorithms, and provide some experimental results derived using this software.
\end{abstract}

{\textbf{Keywords}}: \KS, Online Algorithms, Bijective Analysis

\newpage 
% ADD PRELIMINARIES, LITERATURE REVIEW
\tableofcontents

\newpage

\section{Introduction}
\label{sec:intro}
\subsection{Outline}
\label{sec:out}
In this paper, we consider the \KS problem in a variety of contexts. First, we provide a description of the \KS problem, and an in-depth proof of the $2k-1$ competitiveness of the Work Function Algorithm (WFA). Following this a literature review is provided, focusing on the Unifying Potential for the WFA~\cite{unifyingPotential2021}. We then have an analysis of~\cite{unifyingPotential2021}, and provide some thoughts into a potential extension of this work for caterpillar graphs. Following this, we transition to an analysis of other algorithms focusing on the bijective analysis of these algorithms, and describe a \CC environment provided for practical testing of a variety of these algorithms. We provide some experimental analysis of the algorithms within the testing suite, extending the work done in~\cite{independantStudy2023}.

\subsection{Problem Description}
\label{sec:desc}
An instance of the \KS problem can be described by a metric space $M = (X, d)$, a number of servers $k>1$, and an input sequence $\sigma = (r_1, r_2, r_3, ..., r_n)$, where each $r_i$ corresponds to a point in the metric space. Each of the $k$ servers is assigned an initial starting location within the metric space (generally this is assumed to be the first $k$ requests of the input sequence). Following this, The sequence of reqeusts is processed one at a time. When a request comes in, a given algorithm \textit{ALG} must decide on one of the servers to service the request. It must then move said server from it's current location $x$ to the request $r_i$. This incurs a cost of $c = d(x, r_i)$. The goal is to have \textit{ALG} incur the smallest possible cost while servicing all of the requests in the sequence~\cite{OnlineComp1998}.
\\ \\
There are two major distinctions to be made between different classes of algorithms. The first is the classification of a "lazy" algorithm - one that only moves a server in order to service the current request. Non-lazy algorithms will process a request, and then potentially also move other servers preemptively in order to prepare for future requests. It is important to note that for any non-lazy algorithm that performs well, there is a parallel lazy algorithm that performs just as well, if not better. We can describe this algorithm as follows: suppose we have our non-lazy algorithm, \textit{ALG}. We have the algorithm \textit{LAZY} service requests with the same servers that \textit{ALG} services requests. Suppose that \textit{ALG} moves a server from location $x_1$ to location $x_2$ preemptively, and then later services request $r_i$ with this server. \textit{LAZY} will be servicing $r_i$ with the same server, except it will be moving directly from $x_1$ instead of $x_2$. By the triangle inequality, $d(x_1, r_i) \leq d(x_1, x_2) + d(x_2, r_i)$. By applying this principle throughout the request sequence, we will see that the cost of \textit{LAZY} will be as good if not better than that of \textit{ALG}. This shows us that we can create a lazy algorithm from a non-lazy algorithm by maintaining "ghost" locations for servers in relation to how the non-lazy algorithm would use them. Then, we can determine which server the non-lazy algorithm would use, and then service that location with the correct server~\cite{OnlineComp1998}.
\\ \\
The second major distinction is between "online" and "offline" algorithms. An offline algorithm receives the entire request sequence at once, and so as a result is able to make decisions on what server to use for the current request based off of future requests. In contrast, an online algorithm receives the request one at a time, and so is only able to make decisions based off of past requests, the current server configuration, and the current request. While online algorithms are at a severe disadvantage due to this, real world applications often rely on the performance of these algorithms. Applications range from disk access optimization, such as the two headed-disk problem~\cite{OnlineComp1998}, to police or firetruck servicing.

\subsection{Competitive Analysis}
\label{sec:compAna}
The most popular method for analyzing the performance of online algorithms for a problem is competitive analysis. Here, we assume a "malicious adversary", who is attempting to make our algorithm \textit{ALG} perform as poorly as possible in relation to the performance of the optimal algorithm \textit{OPT}. The malicious adversary is allowed to come up with any finite input, and our competitive ratio is determined by this. If for any finite input, we are able to guaruntee that \textit{ALG} has a cost within a certain ratio $c$ of \textit{OPT} (allowing for a constant additive factor), then we say that our algorithm is $c$-competitive~\cite{OnlineComp1998}. So, if we define $ALG(\sigma)$ as the cost \textit{ALG} incurs while processing request $\sigma$, then we have the following definition: 

\begin{definition}
\label{def:comp}
Algorithm \textit{ALG} is said to be \textbf{\textit{c}-competitive} if for every finite request sequence \s, $ALG(\sigma) \leq c\cdot OPT(\sigma)+\alpha$ for some constant $\alpha$.
\end{definition}

Competitive analysis is in some sense similar to "worst case" analysis, where we try to see how poorly an algorithm will ever perform. It is worth noting that some algorithms such as a greedy algorithm may perform very well on the majority of inputs, but given specific inputs may not be competitive at all. That is, given some value $c$, a finite length input can be found such that the algorithm doesn't satisfy the above definition. 
\\ \\
Additionally, a lower bound of $k$ has been shown for any online algorithm's competitive ratio. This means that if we are looking at the 3-Servers problem, the best competitive ratio an online algorithm can achieve is 3~\cite{OnlineComp1998}.

\section{Algorithms}
\label{sec:algos}

\subsection{Random Algorithm}
\label{sec:rand}
If the request is not currently covered by a server, then the random algorithm $\mathrm{RAND}$ randomly selects one of it's servers, and moves the selected server to the service point. We use this algorithm as a baseline, as we would hope that none of the other algorithms will perform worse than random.

\begin{definition}
    If the request is not currently covered by the configuration, $\mathrm{RAND}$ selects a server uniformly at random, and moves that server to the service point.
\end{definition}

\subsection{Greedy Algorithm}
\label{sec:greedy}
The greedy algorithm is also a computationally inexpensive online algorithm, but which has many practical uses. This algorithm checks the distance that each server would have to travel to get to the service point, and selects the server which would incur the smallest cost. While this algorithm has very good performance for the majority of practical application inputs, it is surprisingly not a competitive algorithm~\cite{OnlineComp1998}. The greedy algorithm is often looked over due to it's non-competitiveness, but still performs very well in practice - especially when compared via other methods~\cite{bij2016, MAXMAX2005}.

\begin{definition}
    If the request is not currently covered by the configuration, $\mathrm{GREEDY}$ selects the server $x$ to move according to the following policy:
    \begin{equation*}
        s = \mathrm{argmin}_{x \in C} d(s, r)
    \end{equation*}
\end{definition}

\subsection{Optimal Algorithm}
\label{sec:OPT}
Any optimal offline algorithm is simply defined as an algorithm that will have the smallest possible cost for every request sequence. Computationally, the fastest implementations leverage a reduction to a Min Cost Max Flow problem, similar to the one described for the $\mathrm{WFA}$ in section~\ref{sec:wfalg}. This reduction can be further studied in~\cite{WFA2009}. This still ends up being a good bit more computationally expensive than the previous algorithms, but is needed to be used as a metric to compare algorithms against, as most algorithms are compared to the optimal when determining their strenghts.

\begin{definition}
    The optimal algorithm $\mathrm{OPT}$ is defined as the offline algorithm that will have the smallest possible cost for every request sequence.
\end{definition}

\subsection{Work-Function Algorithm}
\label{sec:WFA}
The Work-Function Algorithm ($\mathrm{WFA}$) is considered to be one of the most promising algorithms in terms of the competitive ratio, as it has been proven to be $(2k-1)$-competitive, and is believed to be $k$-competitive. Given a certain starting server configuration, current configuration, and previous input request sequence, $\mathrm{WFA}$ balances between which server an $\mathrm{OPT}$ algorithm would likely move and a $\mathrm{GREEDY}$ policy, while also ensuring that $k-1$ of the servers end up in the current server configuration. This final server is then used to service the current request. It is able to do this by computing a \textbf{work function} for the current request, given the previous request sequence and the current server configuration.

\begin{definition}
    The \textbf{work function}, denoted $w_\sigma(C)$ computes the minimum value required to begin in configuration $C_0$, service all requests in $\sigma$, and end with servers in configuration $C$. The work function has similar notation to configurations($w_{\sigma_i}$, $w$, and $w'$), except the work function value on an empty request sequence is denoted $w_\emptyset$, and the work done on the $i$th request sequence is denoted with the abreviated request sequence in the subscript. Additionally, the work function can be computed as follows:
    \begin{equation*}
        \begin{split}
            &w_\emptyset(C) = D(C_0, C) \\
            &w'(C) = \mathrm{min}_{x \in C} \{ w(C - x + r) + d(x, r)\}
        \end{split}
    \end{equation*}
\end{definition}

\begin{definition}
    The $\mathrm{WFA}$ moves the server currently at point $s$ at each step, incurring a cost $d(s,r)$. This server is determined as follows:
    \begin{equation*}
        s = \mathrm{argmin}_{x \in C} \{ w(C-x+r) + d(x,r)\}
    \end{equation*}
\end{definition}

A similar reduction to the $\mathrm{WFA}$ computation can be used to find this server~\cite{WFA2009}. This means that the $\mathrm{WFA}$ must compute a Min-Cost Max-Flow for each request, making it much more expensive than $\mathrm{OPT}$, and all of the other algorithms. Additionally, it is also worth noting that this is not a finite memory algorithm, as the $\mathrm{WFA}$ must remember all of the previous requests~\cite{MAXMAX2005}. 

\subsection{Double Coverage Algorithm}
\label{sec:DC}
The Double Coverage algorithm ($\mathrm{DC}$) is a $k$-competitive algorithm defined only on the line metric space. If the request is to the left of all of our current server locations, then we just move the left most server to service the request. If the request is to the right of all of the current locations, we use the rightmost server. Otherwise, the request will be between two servers. In this case, we move the two closest servers towards the request at the same rate, until one of the servers reaches the request location. A proof of this algorithms competitiveness can be found in~\cite{OnlineComp1998}. It is also important to note that this is not a lazy algorithm, as it does move more than one server at a time. 

\begin{definition}
    If the request is not currently covered by the configuration and the request is further left or further right than any of the servers, then $\mathrm{DC}$ moves the leftmost or rightmost server (respectively) to the service point. Otherwise, $\mathrm{DC}$ selects the closest server on either side of the request $x$ and $y$, and moves them both a distance $p = \mathrm{argmin}_{v \in \{x, y\}} d(v, r)$.
\end{definition}

\subsection{Double Coverage Tree Algorithm}
Additionally, we can generalize the Double Coverage Algorithm to tree metric spaces. 

\begin{definition}
    $\mathrm{DCT}$ does a search for all servers that have a path to the request, where no other server lies on this path. It then moves all servers that satsify this condition towards the request at the same speed, and rechecks the condition each time servers reach a new node. It proceeds in this manner until the request is serviced.
\end{definition}

\subsection{$k$-Centers Algorithm}
\label{sec:KC}
The $k$-Centers algorithm ($\mathrm{KC}$)divides the metric space into $k$ sections, and assigns each server a section of the space to operate in. Implementations of this algorithm can either have the server return to the center of it's operating space after servicing the request in a non-lazy fashion, or just remember the bounds of each operating space, and service each request with the appropriate server~\cite{bij2016}. This algorithm will vary depending on how the metric space is divided, so we only provide an intuitive definition for the line segment metric space.

\begin{definition}
    The $\mathrm{KC}$ algorithm divides the line segment it operates on into $k$ equal sections. Each server is assigned a section to operate in. Whenever a request is not currently covered by a server, the server assigned to the section that the request is in will move to service the request.
\end{definition}

\section{Work Function Algorithm}
\label{sec:wfa}
\subsection{Algorithm}
\label{sec:wfalg}
This section focuses on an in-depth analysis of the WFA. There are two ways of thinking about the WFA's decision-making process. WFA can be thought of as a balancing between retrospective-$OPT$, and a greedy algorithm. It tries to balance between the decisions $OPT$ would have made up until this point, assuming the current request is the last one that will be made, with a greedy algorithm that attempts to move the least distance possible at the current time step. This helps the algorithm maintain a configuration that is similar to $OPT$, without moving servers great distances from the current configuration.

Additionally, $WFA$ can be thought of from how it's implementation works. $WFA$ makes lazy decisions to move servers, so only moves one server at each step. As described previously, the $WFA$ will attempt to determine the server to move which would have the minimum cost for starting in an initial configuration, and ending with $k-1$ servers in the same locations they are currently in. Then, it will service the request with the server that does not stay in the initial configuration.

\subsection{$2k-1$ Proof}
\label{sec:2k1p}
In this section, we follow the proof showing that the $WFA$ is $2k-1$ competitive from~\cite{OnlineComp1998}, along with some in-depth explanatory details. We start by defining the term $w(C)$, the work done to service a request sequence $\sigma_i$ and end in configuration $C$, assuming an initial starting configuration $C_0$. Additionally, we define $w'(C)$ as the work done to service a request sequence $\sigma_{i+1}$ and end in the configuration $C$. It is worth noting a basic property of the work function:

\begin{equation*}
    \label{eq:work}
    w'(C) = min_{x \in C} [w'(C - x + r) + d(r, x)] = min_{x \in C} [w(C - x + r) + d(r, x)]
\end{equation*}

The first part of this says that the work done to service a request sequence \textit{including} the next request $r$ for some configuration $C$ is equal to the minimum work done to service the same request sequence, except on the configuration $C$ without the point $x$ and with the point $r$, plus the distance between $r$ and $x$. The set $C-x+r$ is defined as in~\cite{OnlineComp1998}, where there is only one copy of $r$ in this set, whether or not $r$ was originally in $C$. 

The nice part about this is that we now \textit{know} that $r \in C - x + r$, which isn't necessarily true of $C$. This means that the second part of this equation is true, where we can remove the $r$ from the work function, as it is already included in the set. 

\begin{definition}
    The \textbf{offline pseudocost} $w'(C') - w(C)$ is the work done to service a request sequence $\sigma_{i+1}$ and end in configuration $C'$, minus the work done to service a request sequence $\sigma_i$ and end in configuration $C$.
\end{definition}

The offline pseudocost is a measure of how much work is done to service the next request in the sequence, given that the current configuration is $C$. If we sum this across all requests on sets $C_0$, $C_1$ ... $C_n$ for the configuration of \textit{OPT}, we will see that this will telescope to $w_{\sigma}(C_n) - w_{\emptyset}(C_0)$, where $w_{\theta}(\centerdot)$ is the work done on request sequence $\theta$ to get into a given configuration. This is equal to the cost $OPT(\sigma)$, as we are subtracting the cost to get into the initial configuration $C_0$ from the final work function $w_\sigma(C_n)$.

\begin{definition}
    The \textbf{extended cost} is defined as $max_X \{ w'(X) - w(X)\}$.
\end{definition}

The extended cost represents the worst case configuration we could be in given the next request. That is, the configuration that will maximize the cost incurred to satisfy the next request. The extended cost represents the maximum cost we could incurr by servicing the next request online.

\begin{definition}
    A work function is \textbf{Quasi-Convex} if for all configurations $X$, $Y$, and for all $x \in X$, the following property holds:
    \begin{equation*}
        min_{y \in Y} \{ w(X - x + y) + w(Y - y + x)\} \leq w(X) + w(Y)
    \end{equation*}
\end{definition}

Quasi-Convexity essentially says that for every two configurations, the sum of the costs to get into those configurations is greater than or equal to the \textit{best} cost of swapping \textit{some} point $y\in Y$ with $x$. The key point here is that there will be some minimizing configurations $X$ and $Y$.

\begin{lemma}

\end{lemma}

% \begin{definition}
%     \label{def:off}
%     The \textbf{offline pseudocost} $w' (C') - w(C') is the work done to $

\section{Unifying Potential}
\label{sec:uniPot}
\subsection{Unifying Potential Function Description}
\label{sec:unifyingPotentialDescription}
\subsection{Unifying Potential Function Description}
\label{sec:unifyingPotentialDescription}
\subsection{Unifying Potential Function Description}
\label{sec:unifyingPotentialDescription}
\input{sections/subsections/unifying_potential/unifying_potential.tex}

\subsection{Useful Insights}
\label{sec:usefulInsights}
\input{sections/subsections/unifying_potential/useful_insights.tex}

\subsection{Unifying Potential on the Caterpillar Graph}
\label{sec:cat}
\input{sections/subsections/unifying_potential/caterpillar.tex}

\subsection{Useful Insights}
\label{sec:usefulInsights}
In this section, we provide a couple of useful insights that have been noted while looking at the unifying potential. These range from general properties of work functions, to specific properties for the unifying potential. Additionally, we go over a few of the proofs provided in~\cite{unifyingPotential2021}. 

\subsubsection*{General Work Properties}

\begin{lemma}
    \label{lem:lip}
    Work functions are 1-Lipschitz, i.e. for any work function and any two configurations $X$, $Y$ on any metric space, the following property holds: 
    \begin{equation*}
        w(X) - w(Y) \leq d(X, Y)
    \end{equation*} 
\end{lemma}

\begin{proof}
    This follows directly from the triangle inequality, and is more self explanatory in the following form: 
    
    \begin{equation*}
        w(X) \leq w(Y) + d(X, Y)
    \end{equation*}

    This says that the \textit{worst} value for the work done to get into configuration $X$ would be the work done to get into configuraiton $Y$, plus the cost to transfer between the two configurations.
\end{proof}

\begin{definition}
    It is said that configuration $Y$ \textbf{supports} $X$ if the following property holds:
    \begin{equation*}
        w(X) - w(Y) = d(X,Y)
    \end{equation*}
\end{definition}

\begin{lemma}
    A lazy optimal algorithm $OPT$ has no reason to be in a configuration that is supported by another configuration.
\end{lemma}

\begin{proof}
    Suppose $OPT$ is in a configuraiton $X$ that is supported by another configuration $Y$ after servicing some request $r_i$. Suppose points in $X$ and $Y$ are indexed according to their minimum cost matching, i.e. $x_j$ matches with $y_j$ in the matching. We can define a new offline algorithm, $OPT_1$, which ends in configuration $Y$ after servicing $r_i$. Following this, Whenever $OPT$ services a request $r$ with the $j$th server located at $x_j$, $OPT_1$ will service the request with the same server, located at $y_j$. We know that $d(y_j, r) \leq d(y_j, x_j) + d(x_j, r)$, and $OPT$ has already incurred the cost $d(y_j, x_j)$ (by the definition of support). Therefore, the cost incurred by $OPT_1$ will be less than or equal to that incurred by $OPT$.
\end{proof}

This provides some useful insights, and a new way of looking at this problem. Since at each step, we know that $OPT$ can only be in a limited set of configurations (the set of configurations that are not supported by other configurations), we must only "keep up" with those.

\begin{definition}
    A \textbf{tree graph} is a set of nodes that are connected by edges which is acyclic (No path of unique edges leads back to the starting node), and satisfies the metric property. Additionally, a \textbf{leaf node} is one that is only connected by \textit{one} edge to the rest of the graph.
\end{definition}

\begin{lemma}
    For any metric space $M$ that can be represented as a tree graph, there exists a minimizer $A$ of $r$ with respect to $w$ such that every $x \in A$, $x$ is a leaf node.
\end{lemma}

\begin{proof}
    Suppose $A$ is a minimizer of $r$ with respect to $w$, with $x \in A$ such that $x$ is not a leaf node. We will show that for some leaf node $l$, $A-x+l$ is a minimizer of $r$ with respect to $w$. 

    Because $x$ is not a leaf node, it must be connected to at least two other nodes. This means that there exists some node $y$ that is connected to $x$ by an edge, where $d(x,r) + d(x,y) = d(y,r)$. Therefore, we have the following:

    \begin{equation*}
        \begin{gathered}
            w(A-x+y) - \Sigma_{a \in A-x+y} d(a,r) = w(A-x+y) - \Sigma_{a \in A} d(a,r) + d(x,r) - d(y,r) \\
            = w(A-x+y) - \Sigma_{a \in A} d(a,r) + d(x,r) - d(x,r) - d(x,y)\\
            = w(A-x+y) - \Sigma_{a \in A} d(a,r) - d(x,y) \leq w(A) - \Sigma_{a \in A} d(a,r)
        \end{gathered}
    \end{equation*}

    Where the final inequality holds by eq.~\ref{lem:lip}.
\end{proof}
% \subsubsection*{$k=2$ is $k$ competitive}

% Here, we follow the proof provided in the apendix of~\cite{unifyingPotential2021}.

% \begin{lemma}
%     For any work function $w$ with $k = 2$ and on any metric space, $\Phi(w) = \Phi_{x_1, r} (w)$, and therefore the WFA is 2 competitive.
% \end{lemma}

% \begin{proof}
%     We will show that for any $x_1, x_2 \in M$, $\Phi_{x_1, x_2} (w) \geq min\{ \Phi_{x_1, r}, \Phi_{x_2, r}\}$, proving the lemma.
% \end{proof}

\subsection{Unifying Potential on the Caterpillar Graph}
\label{sec:cat}
\begin{definition}
    A \textbf{Multiray space} is a tree metric space $(X, d)$ wich consists of a central node $c$, and rays that are rooted at $c$. Every node $x \in X - c$ is connected to \textit{at most} two other nodes. This can be thought of as multiple lines with endpoints joined at $c$.
\end{definition}

\begin{figure}[H]
    \centering
    \includegraphics[width=0.5\textwidth]{images/multiray.png}
    \caption{An example multiray space}
\end{figure}

\begin{definition}
    A \textbf{General Caterpillar Graph} is a tree metric space $(X, d)$ consisting of a central line path, and multiple legs extending from the central path. Each central path node can have 0, 1, or 2 legs attached to it. Each leg is a line segment. The multiray space can be thought of as a compression of a general caterpillar graph, where the distance between nodes on the central line path is 0.
\end{definition}

\begin{figure}[H]
    \centering
    \includegraphics[width=0.5\textwidth]{images/generalCaterpillar.png}
    \caption{An example general caterpillar graph}
\end{figure}

\begin{definition}
    A \textbf{Reduced Caterpillar Graph} is a General Caterpillar Graph where the distances between nodes on the central path are all equal to 1. Additionally, all body nodes have exactly 2 legs attached to them, and each leg has exactly 1 node on it. The reduced caterpillar graph is a special case of the general caterpillar graph.
\end{definition}

\begin{figure}[H]
    \centering
    \includegraphics[width=0.5\textwidth]{images/reducedCaterpillar.png}
    \caption{An example reduced caterpillar graph}
\end{figure}

In this section, we will be primarily considering the smallest possible reduced caterpillar graph, consisting of 6 nodes. A unifying potential proof has been shown for multiray spaces, and we will present some notes on the extension of this proof to this highly reduced caterpillar graph. It is worth noting that the $WFA$ can already be shown to be $k$ competitive on this space with a variety of proofs. That is, 2 servers have been shown to be 2 competitive on general metric spaces, and 3 servers have been shown to be 3 competitive on trees. 4 servers on this space is a metric space with $k-2$ points, and 5 servers is a metric space of $k-1$ points, which has yet again been shown to be $k$ competitive~\cite{unifyingPotential2021}.



\subsection{Useful Insights}
\label{sec:usefulInsights}
In this section, we provide a couple of useful insights that have been noted while looking at the unifying potential. These range from general properties of work functions, to specific properties for the unifying potential. Additionally, we go over a few of the proofs provided in~\cite{unifyingPotential2021}. 

\subsubsection*{General Work Properties}

\begin{lemma}
    \label{lem:lip}
    Work functions are 1-Lipschitz, i.e. for any work function and any two configurations $X$, $Y$ on any metric space, the following property holds: 
    \begin{equation*}
        w(X) - w(Y) \leq d(X, Y)
    \end{equation*} 
\end{lemma}

\begin{proof}
    This follows directly from the triangle inequality, and is more self explanatory in the following form: 
    
    \begin{equation*}
        w(X) \leq w(Y) + d(X, Y)
    \end{equation*}

    This says that the \textit{worst} value for the work done to get into configuration $X$ would be the work done to get into configuraiton $Y$, plus the cost to transfer between the two configurations.
\end{proof}

\begin{definition}
    It is said that configuration $Y$ \textbf{supports} $X$ if the following property holds:
    \begin{equation*}
        w(X) - w(Y) = d(X,Y)
    \end{equation*}
\end{definition}

\begin{lemma}
    A lazy optimal algorithm $OPT$ has no reason to be in a configuration that is supported by another configuration.
\end{lemma}

\begin{proof}
    Suppose $OPT$ is in a configuraiton $X$ that is supported by another configuration $Y$ after servicing some request $r_i$. Suppose points in $X$ and $Y$ are indexed according to their minimum cost matching, i.e. $x_j$ matches with $y_j$ in the matching. We can define a new offline algorithm, $OPT_1$, which ends in configuration $Y$ after servicing $r_i$. Following this, Whenever $OPT$ services a request $r$ with the $j$th server located at $x_j$, $OPT_1$ will service the request with the same server, located at $y_j$. We know that $d(y_j, r) \leq d(y_j, x_j) + d(x_j, r)$, and $OPT$ has already incurred the cost $d(y_j, x_j)$ (by the definition of support). Therefore, the cost incurred by $OPT_1$ will be less than or equal to that incurred by $OPT$.
\end{proof}

This provides some useful insights, and a new way of looking at this problem. Since at each step, we know that $OPT$ can only be in a limited set of configurations (the set of configurations that are not supported by other configurations), we must only "keep up" with those.

\begin{definition}
    A \textbf{tree graph} is a set of nodes that are connected by edges which is acyclic (No path of unique edges leads back to the starting node), and satisfies the metric property. Additionally, a \textbf{leaf node} is one that is only connected by \textit{one} edge to the rest of the graph.
\end{definition}

\begin{lemma}
    For any metric space $M$ that can be represented as a tree graph, there exists a minimizer $A$ of $r$ with respect to $w$ such that every $x \in A$, $x$ is a leaf node.
\end{lemma}

\begin{proof}
    Suppose $A$ is a minimizer of $r$ with respect to $w$, with $x \in A$ such that $x$ is not a leaf node. We will show that for some leaf node $l$, $A-x+l$ is a minimizer of $r$ with respect to $w$. 

    Because $x$ is not a leaf node, it must be connected to at least two other nodes. This means that there exists some node $y$ that is connected to $x$ by an edge, where $d(x,r) + d(x,y) = d(y,r)$. Therefore, we have the following:

    \begin{equation*}
        \begin{gathered}
            w(A-x+y) - \Sigma_{a \in A-x+y} d(a,r) = w(A-x+y) - \Sigma_{a \in A} d(a,r) + d(x,r) - d(y,r) \\
            = w(A-x+y) - \Sigma_{a \in A} d(a,r) + d(x,r) - d(x,r) - d(x,y)\\
            = w(A-x+y) - \Sigma_{a \in A} d(a,r) - d(x,y) \leq w(A) - \Sigma_{a \in A} d(a,r)
        \end{gathered}
    \end{equation*}

    Where the final inequality holds by eq.~\ref{lem:lip}.
\end{proof}
% \subsubsection*{$k=2$ is $k$ competitive}

% Here, we follow the proof provided in the apendix of~\cite{unifyingPotential2021}.

% \begin{lemma}
%     For any work function $w$ with $k = 2$ and on any metric space, $\Phi(w) = \Phi_{x_1, r} (w)$, and therefore the WFA is 2 competitive.
% \end{lemma}

% \begin{proof}
%     We will show that for any $x_1, x_2 \in M$, $\Phi_{x_1, x_2} (w) \geq min\{ \Phi_{x_1, r}, \Phi_{x_2, r}\}$, proving the lemma.
% \end{proof}

\subsection{Unifying Potential on the Caterpillar Graph}
\label{sec:cat}
\begin{definition}
    A \textbf{Multiray space} is a tree metric space $(X, d)$ wich consists of a central node $c$, and rays that are rooted at $c$. Every node $x \in X - c$ is connected to \textit{at most} two other nodes. This can be thought of as multiple lines with endpoints joined at $c$.
\end{definition}

\begin{figure}[H]
    \centering
    \includegraphics[width=0.5\textwidth]{images/multiray.png}
    \caption{An example multiray space}
\end{figure}

\begin{definition}
    A \textbf{General Caterpillar Graph} is a tree metric space $(X, d)$ consisting of a central line path, and multiple legs extending from the central path. Each central path node can have 0, 1, or 2 legs attached to it. Each leg is a line segment. The multiray space can be thought of as a compression of a general caterpillar graph, where the distance between nodes on the central line path is 0.
\end{definition}

\begin{figure}[H]
    \centering
    \includegraphics[width=0.5\textwidth]{images/generalCaterpillar.png}
    \caption{An example general caterpillar graph}
\end{figure}

\begin{definition}
    A \textbf{Reduced Caterpillar Graph} is a General Caterpillar Graph where the distances between nodes on the central path are all equal to 1. Additionally, all body nodes have exactly 2 legs attached to them, and each leg has exactly 1 node on it. The reduced caterpillar graph is a special case of the general caterpillar graph.
\end{definition}

\begin{figure}[H]
    \centering
    \includegraphics[width=0.5\textwidth]{images/reducedCaterpillar.png}
    \caption{An example reduced caterpillar graph}
\end{figure}

In this section, we will be primarily considering the smallest possible reduced caterpillar graph, consisting of 6 nodes. A unifying potential proof has been shown for multiray spaces, and we will present some notes on the extension of this proof to this highly reduced caterpillar graph. It is worth noting that the $WFA$ can already be shown to be $k$ competitive on this space with a variety of proofs. That is, 2 servers have been shown to be 2 competitive on general metric spaces, and 3 servers have been shown to be 3 competitive on trees. 4 servers on this space is a metric space with $k-2$ points, and 5 servers is a metric space of $k-1$ points, which has yet again been shown to be $k$ competitive~\cite{unifyingPotential2021}.



\section{Experimental Analysis}
\label{sec:bijAna}

\subsection{Analysis Methods}
\label{sec:bijAnalysisDesc}
In this section, we present a few additional methods for analysis on algorithms other than competitive analysis. These methods are useful for comparing algorithms to each other, and can be used to demonstrate an algorithms efficiency outside of the "worst-case" considered by competitive analysis. Competitive analysis considers all requests of finite length. For each of these methods described below, we take an experimental-based approach. That is, each analysis method will be considerind the performance of an algorithm on a given request sequence length, and so can therefore be directly calculated in practice.

\subsubsection*{Direct Analysis}
\label{sec:Direct}
Direct analysis is a similar techinque to competitive analysis, in that we directly compare the performance of \textit{ALG} to the performance of \textit{OPT} on a request. Rather than looking across all request sequences of any finite length, we determine a specified length for our request sequence. Then, we look directly at the ratio of \textit{ALG}'s to \textit{OPT}'s performance for each input, and take the largest such ratio. It is important to note that as the length of our input approaches infinity, our direct analysis ratio will aproach the competitive ratio of our algorithm.

\begin{definition}
    \label{def:direct}
    Algorithm \textit{ALG} has a direct analysis ratio of $c$ if for every input \s of length $n$, $ALG(\sigma) \leq c\cdot OPT(\sigma)$.
\end{definition}

\subsubsection*{Max/Max Ratio}
\label{sec:MaxMax}
For the Max/Max ratio, we are comparing the worst case of each algorithm to each other. In practice, this makes sense to do on finite sets of input sequences. Here, we will take the highest cost of \textit{ALG} and the highest cost of \textit{OPT}, and this ratio will be our performance metric~\cite{MAXMAX2005}. 

\subsubsection*{Bijective Analysis}
\label{sec:Bij}
The bijective ratio is similar to using direct analysis, except we allow for a bijection between the two data sets. So, we look at the input space of all inputs of length $n$, denoted $I_n$. If there exists a bijetion $\pi:I_n \rightarrow I_n$ such that $ALG(\sigma) \leq c\cdot OPT(\pi(\sigma))$, then  we say that the bijective ratio between \textit{ALG} and \textit{OPT} is $c$~\cite{bij2016}. Therefore, we obtain def.~\ref{def:bij}. 

\begin{definition}
    \label{def:bij}
    For a given input space $I_n$, we say that algorithm \textit{A} has bijective ratio $c$ with respect to algorithm \textit{B} if there exists a bijection $\pi:I_n \rightarrow I_n$ such that $A(\sigma) \leq c\cdot B(\pi(\sigma))$, $\forall \sigma \in I_n$. 
\end{definition}

It is important to note that this definition doesn't only compare the ratio of an algorithm to the optimal, but can also be used to compare between two different online algorithms. This can be used to prove an online algorithms optimality~\cite{bij2016}, and allows for some interesting test-case results found in sec.~\ref{sec:analysis}.

\subsection{\CC Implementations}
\label{sec:implementation}
In this section, we describe the software package developed as part of this work. The package is written in C++, and has various levels of performance capabilities. It can additionally be run using the Lemon Graph Library~\cite{lemon}, or with purely internal data structure implementations. The various implementations allow users to run the various algorithms described in sec.~\ref{sec:algos} either with a single thread, or under a producer-consumer framework for parallel processing. Additionally, a finite memory implementation is provided, and a memoization approach on the input sequence is implemented allowing for increased efficiency. Finally, example code for running the software on a High Throughput Cluster using Condor~\cite{htcondor} is provided.

\subsubsection*{Basic structures}

The most basic structure in the software package is the \texttt{Mspace} class. This class uses a \texttt{std::vector} to store the graph as an adjacency matrix of distances between nodes. Functions such as \texttt{setSize}, \texttt{getSize}, \texttt{getDistance}, and \texttt{setDistance} allow for easy manipulation of the metric space.

\texttt{getInput} and \texttt{writeOutput} classes are written to help maintain RAII standards, and allow for easy reading and writing of input and output files. 

An abstract \texttt{Alg} class is provided as a framework for each of the various algorithms to build off of. This class defines all basic functionality for an algorithm to run, and has a virtual function \texttt{run} that must be implemented by each algorithm, which takes an input request sequence and returns the cost incurred by the algorithm. It contains a \texttt{Mspace} object, and stores the current configuration of the servers in two forms: a \texttt{config} vector which stores the integer location of each server, as well as a \texttt{coverage} vector that stores whether or not each location in the metric space is covered by a server. This allows for efficient checking of whether or not a location is covered by a server, as well as efficient access to where the servers are located. \texttt{setServers} and \texttt{setGraph} functions are provided to allow for easy initialization of the algorithm, and a \texttt{moveServer} function is provided for easy manipulation of the server configuration.

\subsubsection*{Algorithm Implementations}

The \texttt{Alg} class is inherited by each of the various algorithm implementations. Each algorithm has a \texttt{run} function that takes an input request sequence and returns the cost incurred by the algorithm. Algorithms \texttt{doubleCoverageAlg}, \texttt{lazydoubleCoverageAlg} and \texttt{KCentersAlg} only work on line metric spaces, and so have a check to insure that the input metric space is a line. The \texttt{doubleCoverageTreeAlg} is the extension of the \texttt{doubleCoverageAlg} to work on trees, but has no check to ensure that the given metric space is a tree. Additional algorithm implementations designed to work on generic metric spaces include a \texttt{randomAlg} that randomly chooses which server to move, \texttt{greedyAlg}, \texttt{WFAlg}, and \texttt{optAlg} which operates on the entire request sequence at once. \texttt{WFAlg} uses a min cost flow problem formulation, as described in sec.~\ref{sec:mcfp}. \texttt{optAlg} uses a similar formulation, which is described in~\cite{mcfp2011}. Solvers for these are provided in two classes, \texttt{mcfp} and \texttt{lemon\_mcfp}. The \texttt{mcfp} uses a custom graph implementation and a provided solver, allowing for a stand-alone build. the \texttt{lemon\_mcfp} solver constructs the graph within the Lemon Graph Library data structures, and uses the Lemon solvers allowing for a 6x speedup~\cite{lemon}. Both of these are implemented in spereate classes to once again help maintain RAII standards. While the most recent efficient implementation design does not use these classes for the algorithms, they are still provided for ease of understnading.

\subsubsection*{Analysis Implementations}

The most basic mechanism for running the algorithsm is provided in the \texttt{csv\_parser} file. This provides methods for reading the input and output file names, and then reading and constructing the metric space from the input csv file. For each input, the various algorithm are run and their costs are saved in a vector. The output costs are then written to a csv file along with their associated algorithm names and the request sequence they correspond to. This implementation struggles due to its sequential nature, and unbounded memory usage.

The next implementaiton, \texttt{csv\_parser\_parralel}, uses \texttt{OpenMP}~\cite{openmp08} to parallelize the algorithm runs across multiple cores. This requires \texttt{OpenMP} to be installed, and for the program to be compiled with the appropriate flags. This is a simple parallelization of the above implementation, but still struggles from unbounded memory usage.

The \texttt{csv\_parser\_parralel2} implements a producer-consumer framework with the thread strucuture provided by the \texttt{C++11} standard. This implementation launches multiple consumer threads which individually run each request sequence, and then push results to a \texttt{buffer} class implementaition. A consumer thread then reads from the buffer and writes the results to the output file. This allows for a smaller amount of memory usage, and a more efficient use of the available resources.


The final implementation uses a memoization approach to increase efficiency. This is possisble due to the nature of the \KS problem. For a request sequence of length $i$, any online algorithm $ALG$ will have the exact same result (with the exception of \texttt{randAlg}) for the first $i-1$ inputs, regardless of the $i$th input. this means that for a metric space consisting of 10 points, we are computing costs for request sequences of length $i-1$ 10 times more than we actually need to when following the previous approaches. The \texttt{all\_efficient} implementation saves multiple configuration / request sequence states for each algorithm in memory, allowing it to reuse the results of previous runs. It additionally still utilizes the producer-consumer framework implemented in \texttt{csv\_parser\_parralel2}. This allows for a significant speedup in the computation of the costs for each algorithm, with the use of slightly more memory.

Additionally, example \texttt{.sub} scripts are provided for use on a High Throughput Cluster using Condor. This allows for easy submission of multiple jobs to the cluster~\cite{htcondor}. 

\subsubsection*{Input Generation}

There are various files provided dedicated to generating different inputs in a \texttt{csv} file format. Each file behaves very similarly, with variance in the type of metric space generated. Unfortunately, some files may be outdated and generate additional information that is hard-coded in the more recent analysis implementations. All of these assume unit lengths between any two connected nodes. The \texttt{generateInputLine} and \texttt{generateInputCircle} are the most basic input generation mechanisms, which generate metric spaces for lines and circles. These both have the parameters hard-coded in the files, and must be manually modified and recompiled to get various numbers of points on the line and various numbers of servers.

The \texttt{generateStar} and \texttt{generateStar1} files each generate Multiray spaces, with the \texttt{generateStar1} having the restriction that each node other than $c$ is connected \textit{only} to $c$, with a unit length. These must also have the parameters modified within the files.

Finally, the \texttt{generateInputCaterpillar} file will generate a reduced caterpillar graph as described above. It takes three parameters as input when run as an exectuable: the number of nodes on the central path, the number of requests per request sequence, and the number of servers to run with. 

It is worth noting that for the most advanced analysis methods, some variables such as the input request sequence lengths and number of servers are hard-coded within the analysis tool. Earlier versions take these as input along with the metric space.

\subsection{Experimental Results}
\label{sec:exp}
\begin{table}[!htb]
    \begin{minipage}{.5\linewidth}
      \centering
      \begin{tabular}{|c|c|c|c|c|c|}
        \hline
        Input Number & Space Type & Space Size & Servers & Input Length & Number of Inputs\\
        \hline
        1 & Line & 6 & 2 & 8 & ALL \\
        \hline
        2 & Line & 8 & 2 & 7 & ALL \\
        \hline
        3 & Line & 8 & 3 & 7 & ALL\\
        \hline
        4 & Line & 10 & 3 & 8 & ALL\\
        \hline
        5 & Line & 10 & 3 & 20 & 10,000,000\\
        \hline
        6 & Line & 10 & 3 & 100 & 1,000,000\\
        \hline 
        7 & Circle & 6 & 2 & 8 & ALL\\
        \hline
        8 & Circle & 8 & 3 & 8 & ALL\\
        \hline
        9 & Circle & 10 & 2 & 6 & ALL\\
        \hline
        10 & Circle & 10 & 3 & 6 & ALL\\
        \hline
        11 & Circle & 20 & 3 & 6 & ALL \\
        \hline
    \end{tabular}
        \caption{Tests}
        \label{tab:tests}
    \end{minipage}%
    \\
    \begin{minipage}{.5\linewidth}
        \centering
        \begin{tabular}{|c|c|c|c|}
          \hline
          Input Number & GRE/OPT Bij. & WFA/OPT Bij. & WFA/GRE Bij.\\
          \hline
          1 & 3/2 & 3/2 & 4/3\\
          \hline
          2 & 3/2 & 1.55 & 4/3\\
          \hline
          3 & 3/2 & 1.6 & 4/3 \\
          \hline
          4 & 3/2 & 1.59 & 4/3 \\
          \hline
          5 & 1.319 & 1.308 & 1.03 \\
          \hline
          6 & 1.447 & 1.486 & 1.2 \\
          \hline
          7 & 3/2 & 3/2 & 7/6 \\
          \hline
          8 & 3/2 & 3/2 & 4/3 \\
          \hline
          9 & 3/2 & 3/2 & 7/6 \\
          \hline
          10 & 3/2 & 3/2 & 4/3 \\
          \hline
          11 & 3/2 & 3/2 & 4/3 \\
          \hline

      \end{tabular}
          \caption{Results}
          \label{tab:results}
    \end{minipage} 
\end{table}

While data was also collected for bijective ratios between the other algorithms, as well as Max/Max ratios and direct analysis ratios, the data presented is the most notable. $GREEDY$ has already been show to be non-competitive, and competitive ratios for $DC$ and $KC$ have already been proven. For $WFA$, a competitive ratio of $2k-1$ has been proven, and no input with competitive ratio worse than $k$ has been found. This makes the Direct Analysis rather uninteresting. Additionally, the Max/Max ratio on small inputs such as these doesn't seem to provide much insight into patterns, or the true performance of the algorithms.

The first thing to note is that running these algorithms on a small subset of random inputs of a certain length seems to provide little to no insight into the algorithms true performace. Both $GREEDY$ and $WFA$ perform well - if not optimally - on a large subset of the inputs. As our input lengths increase, the number of possible inputs increases exponentially. To demonstrate the phenomena we observe, suppose for example that $GREEDY$ and $WFA$ perform optimally on 90\% of inputs, and we are able to run 1,000,000,000 inputs of any length on a metric space with 10 points in reasonable time (this is not true, as our computation time increase drastically with regard to input length due to the WFA). So, wecan run all inputs of length 9 within this time. If we want to run inputs of length 10, we are only able to run 10\% of the inputs. If we want to run inputs of length 12, we can only run 0.1\% of the inputs! The probability that all of these inputs will land within the 90\% where the algorithms perform optimally increases exponentially as our input length increases, making our bijective ratio approach 1.

Therefore, we can only start to draw interesting conclusions from the input parameters where we can run any possible input sequence.

The final interesting result is with respect to the comparison between $WFA$ and $GREEDY$. The bijection in the other direction is not shown, as it is always 1 (meaning that $GREEDY$ always performs as well if not better than $WFA$, bijectively). The fact that the bijective ratio shown is not 1 suggests some interesting results. If both of these bijections could be proven, then this would show that it is better to use $GREEDY$ rather than $WFA$ (a computationally much more expensive algorithm), across a uniform distribution of inputs. For every input where $GREEDY$ performs poorly, there would be an input where $WFA$ performs just as poorly if not worse.  

\section{Conclusion}
\label{sec:con}
In this work, we analyze the \KS problem and provide descriptions for various algorithms. We primarily focus on the most common form of analysis for the \KS problem, competitive analysis. We provide an overview of various theorems and proofs taken from~\cite{OnlineComp1998}, and additionally go over a fast min-cut max-flow formulation for computing the $\mathrm{WFA}$ movement~\cite{mcfp2011}. We then transition to focusing on the Unifying Potential~\cite{unifyingPotential2021}, overviewing a few of their definitions and proofs. We attempt to extend their methods for Multiray Spaces to Reduced Caterpillar Graphs. Finally, we describe various other analysis methods used in the context of this problem, and provide an open-source software framework for testing and comparing these algorithms. The software package can be found here: \url{https://github.com/StefanCaldararu/KServers}. We finally provide some experimental results derived using this software, and a brief description of these results.


\bibliographystyle{ieeetr}
\bibliography{refs.bib}


\end{document}          